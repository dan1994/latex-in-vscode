\documentclass{article}

	\usepackage[english]{babel}
	\usepackage{xspace}
	\usepackage{graphicx}
	\usepackage{hyperref}

	\newcommand{\latex}{\LaTeX\xspace}

\begin{document}

\title{
	A Guide to \latex Editing in Visual Studio Code\\
	\large Written in \latex with Visual Studio Code
}
\author{Dan Arad}
\maketitle

\section{Introduction}
This guide's purpose is to help setup a comfortable and efficient editing of \latex documents using Visual Studio Code (VSCode from now on). This also includes the ability to easily work with git version control, and keep generated files out of the way.\\
VSCode\footnotemark[0] is a powerful light-weight editor developed and maintained by Microsoft. The editor boasts many extensions and built-in git support, and is widely used. The editor is free and open source. Among its many extension, there is one called \emph{LaTeX Workshop}\footnotemark[1].\\
To me VSCode is the first choice for editing any kind of code, and for that reason I chose to see how well will it preform as a \latex editor. I was not dissapointed with the result.

\footnotetext[0]{Homepage: https://code.visualstudio.com/}
\footnotetext[1]{Extension Site: https://marketplace.visualstudio.com/items?itemName=James-Yu.latex-workshop}
\end{document}